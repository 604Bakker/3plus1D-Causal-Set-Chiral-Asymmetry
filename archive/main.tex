% main.tex
\documentclass[11pt,oneside]{article}

% Encoding & fonts
\usepackage[utf8]{inputenc}
\usepackage[T1]{fontenc}
\usepackage{lmodern}

% Math & symbols
\usepackage{amsmath,amssymb,amsthm}

% Figures, tables
\usepackage{graphicx}
\usepackage{booktabs}
\usepackage{float}
\usepackage{caption}
\usepackage{subcaption}

% Formatting
\usepackage[margin=1in]{geometry}
\usepackage[numbers,sort&compress]{natbib}

% Hyperlinks 
\usepackage{hyperref}

\title{
\bf Emergent Chiral Asymmetry in 3+1D Causal Sets \\[4pt]
from Dirac--Kähler Fermions with Parity-Biased Sprinklings
}

\author{
  Greg Bakker\\
  Independent Researcher\\
  \texttt{gsbbakr@gmail.com}\\
  \url{https://github.com/604Bakker}
}

\date{25 November 2025}

\begin{document}
\maketitle

\begin{abstract}
I present numerical evidence that random $3{+}1$-dimensional causal sets, when
sprinkled with a mild parity-violating bias, undergo a sharp transition to a
phase with a large, topologically robust chiral index. Using the Dirac--Kähler
discretization with a minimal Wilson term, I find that the lowest modes of
$i\gamma_5 D$ develop an excess of $\sim 40$--$50$ zero modes of uniform
handedness once the sprinkling bias exceeds $r \gtrsim 0.11$ and the
discreteness scale exceeds $\varepsilon \gtrsim 0.35$. The effect is stable in
the range $N = 6000$--$8000$ points and extends the chiral plateau previously
observed in $2{+}1$ dimensions. The emergent asymmetry requires no gauge fields,
no Higgs sector, and no continuum limit: it is generated purely by the interplay
between causal-set microstructure and the Dirac--Kähler operator. These results
suggest that discrete spacetime alone may support nontrivial chiral structure,
and I outline connections to potential phenomenology.
\end{abstract}

\section{Introduction}

The origin of chiral asymmetry in fundamental interactions remains a central
problem at the interface of particle physics and quantum gravity. In the
continuum Standard Model, fermion chirality is imposed through gauge
representations and the Higgs mechanism. It is natural to ask whether such
asymmetry could arise instead from microscopic geometric or topological structure
of spacetime itself.

Causal-set theory provides a minimal framework for discrete Lorentzian geometry:
spacetime is replaced by a locally finite poset with order representing causal
precedence \citep{Bombelli1987, Sorkin2003}. While kinematics is well defined,
dynamical and matter-theoretic aspects remain challenging. In prior work, a
striking phenomenon was observed in $2{+}1$ dimensions: with Dirac--Kähler (DK)
fermions, a small parity-violating deformation of the sprinkling distribution
produces a saturated ``chiral plateau'' with dozens of zero modes of identical
handedness \citep{Bakker2025_2p1}. 

In this work I extend that analysis to $3{+}1$ dimensions. I show that the
same mechanism leads to an abrupt topological transition yielding a large,
stable chiral index. Remarkably, the effect persists with no reliance on
continuum limits, gauge fields, or fine-tuning. It is a purely geometric
phenomenon emerging from the discrete structure of the causal set.

\section{Causal-Set Construction and Parity Bias}

I generate causal sets via Poisson sprinklings into a $3{+}1$-dimensional
Minkowski region with volume $V = N$, following standard practice
\citep{Henson2009}. To introduce a controlled parity-violating deformation,
each point $(x,y,z)$ is rotated in the $(x,y)$ plane by angle $r\pi$:
\begin{equation}
 (x,y,z) \;\longrightarrow\; (x\cos(r\pi) - y\sin(r\pi),\,
                              x\sin(r\pi) + y\cos(r\pi),\,
                              z).
\end{equation}
The parameter $r$ thus tunes a left- or right-handed bias in the spatial
distribution. The causal relation is defined by timelike separation, and links
are retained for pairs separated by proper time less than a discreteness scale
$\varepsilon$.

The Dirac--Kähler operator is constructed from signed incidence matrices of
simplices (vertices, links, faces and tetrahedra), following
\citep{Rocchini2023}. A minimal Wilson term is added to suppress fermion
doublers. The chiral index is defined as the difference $n_+ - n_-$ of positive
and negative near-zero eigenmodes of $i\gamma_5 D$, computed over the lowest
140 eigenvalues.

\section{Results}

\subsection{Phase Transition}

Figure~\ref{fig:cliff} shows the chiral index as a function of discreteness
scale $\varepsilon$ for several values of the parity-bias parameter $r$. A
sharp transition is evident: for $r \lesssim 0.11$, the index is strictly zero,
while for $r \gtrsim 0.11$ and $\varepsilon \gtrsim 0.35$, the system develops
a macroscopic negative index of approximately $-45$.

\begin{figure}[htbp]
  \centering
  \includegraphics[width=0.95\textwidth]{3plus1_cliff_plot.png}
  \caption{Chiral index as a function of discreteness scale $\varepsilon$
  for several bias strengths $r$, with $N=6000$--$8000$. A robust negative
  plateau with magnitude $\sim 40$--$50$ emerges above the critical surface
  $r \gtrsim 0.11$, $\varepsilon \gtrsim 0.35$.}
  \label{fig:cliff}
\end{figure}

Representative points in the chiral phase are listed in Table~\ref{tab:phase}.
Fluctuations over multiple realizations are modest ($\lesssim 20\%$), indicating
a topologically stabilized phase analogous to the $2{+}1$-dimensional chiral
plateau.

\begin{table}[htbp]
\centering
\begin{tabular}{ccccc}
\toprule
$r$ & $\varepsilon$ & $N$ & index & trials \\
\midrule
0.40 & 0.50 & 8000 & $-38.1 \pm 8.6$ & 40 \\
0.40 & 0.50 & 8000 & $-51$          & single \\
0.275 & 0.501 & 6000 & $-46.0 \pm 4.7$ & 40 \\
0.220 & 0.501 & 6000 & $-46.0 \pm 4.7$ & 40 \\
\bottomrule
\end{tabular}
\caption{Representative points in the chiral phase. The plateau persists across
$r=0.11$--$0.44$ and $\varepsilon \gtrsim 0.35$.}
\label{tab:phase}
\end{table}

Reversing the sign of $r$ reverses the sign of the asymmetry, showing that the
effect is controlled by handedness rather than numerical asymmetry.

\section{Discussion}

The emergence of a large chiral index in this setup is topological in origin:
the Dirac--Kähler operator detects an imbalance in oriented $3$-simplices induced
by the parity-biased sprinkling. Because the DK operator is built directly from
the incidence structure of the causal set, the chiral index reflects a purely
geometric aspect of the underlying poset.

I emphasize several features:
\begin{itemize}
  \item \textbf{Absence of continuum tuning.} The asymmetry appears at finite
        $N$ and finite $\varepsilon$, without requiring continuum limits.
  \item \textbf{Topological robustness.} Index fluctuations are small even in
        finite systems, suggesting that the effect should survive coarse-graining.
  \item \textbf{Dimensional shift.} The critical curve shifts upward relative to
        the $2{+}1$-dimensional case, consistent with dilution of an effective
        Chern--Simons-like term in higher dimensions.
\end{itemize}

The results point to a potentially rich interplay between discrete Lorentzian
geometry, topological invariants of incidence complexes, and fermionic structure.

\section{Outlook}

Several directions follow naturally:

\begin{itemize}
  \item \textbf{Analytic characterization.} An explicit topological index theorem
        for DK fermions on causal sets may be within reach.
  \item \textbf{Continuum correspondence.} It is important to determine whether
        the emergent index corresponds to a continuum chiral anomaly or effective
        Chern--Simons density.
  \item \textbf{Matter coupling.} Introducing gauge fields or scalar fields may
        clarify whether the mechanism can seed chiral structure resembling that
        of the Standard Model.
  \item \textbf{Phenomenology.} Connections between parity-odd causal-set microstructure
        and isotropic but parity-odd backgrounds (e.g.\ as proposed in analyses
        of isotope-shift anomalies) warrant investigation.
\end{itemize}

\section*{Data and Code Availability}

All simulation code, parameters, and data for this work are available at
\url{https://github.com/604Bakker/3-1D-entropic-foam}. A Zenodo archive with DOI
will accompany the final release.

\begin{acknowledgments}
The author thanks the causal-set research community for helpful discussions and
the open-source scientific Python ecosystem for tools enabling large-scale
computations on modest hardware.

Certain aspects of code implementation, debugging, and text polishing benefited
from interactive assistance provided by large language models (including Grok by
xAI and ChatGPT by OpenAI). All scientific ideas, physical interpretation, and
final conclusions are solely the responsibility of the author.
\end{acknowledgments}

\bibliographystyle{unsrtnat}
\bibliography{references}

\end{document}
