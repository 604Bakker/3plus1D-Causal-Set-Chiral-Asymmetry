% main.tex
\documentclass[11pt,oneside]{article}

% Encoding & fonts
\usepackage[utf8]{inputenc}
\usepackage[T1]{fontenc}
\usepackage{lmodern}

% Math & symbols
\usepackage{amsmath,amssymb,amsthm}

% Figures, tables
\usepackage{graphicx}
\usepackage{booktabs}
\usepackage{float}
\usepackage{caption}
\usepackage{subcaption}

% Formatting
\usepackage[margin=1in]{geometry}
\usepackage[numbers,sort&compress]{natbib}

% Hyperlinks 
\usepackage{hyperref}

\title{
\bf Emergent Chiral Asymmetry in 3\,+\,1D Causal Sets \\[4pt]
from Dirac--Kähler Fermions with Parity-Biased Sprinklings
}

\author{
  Greg Bakker\\
  Independent Researcher\\
  \texttt{gsbbakr@gmail.com}\\
  \url{https://github.com/604Bakker}
}

\date{25 November 2025}

\begin{document}
\maketitle

% ---------------------------------------------------------------
% FIRST-PAGE DATA AVAILABILITY NOTE
% ---------------------------------------------------------------
\noindent\textbf{Data Availability:}
All code and data associated with this study are permanently archived at Zenodo:\
\url{https://doi.org/10.5281/zenodo.17714411}\\[1em]



% ---------------------------------------------------------------
% ABSTRACT
% ---------------------------------------------------------------
\begin{abstract}
I present numerical evidence that random $3{+}1$-dimensional causal sets, when
sprinkled with a minimal parity-violating bias, undergo a transition to a phase
with a large, topologically robust chiral index. Using the Dirac--Kähler
discretization with a Wilson term, I compute near-zero eigenmodes of
$i\gamma_5 D$ for Poisson sprinklings of size $N=6000$ and observe a stable
asymmetry of $\mathcal{O}(40$--$50)$ between positive and negative low-lying
eigenvalues over a broad region of parameter space. The effect is driven
primarily by the parity-bias parameter and persists without any continuum limit,
gauge fields, or Higgs sector. These results demonstrate that discrete spacetime
alone can support nontrivial chiral structure and provide a concrete microscopic
mechanism for parity-odd vacuum structure.
\end{abstract}

\section{Introduction}

The origin of fermion chirality remains a central problem connecting particle
physics, topology, and quantum gravity. In the continuum Standard Model,
chirality is enforced by gauge representations and the Higgs mechanism. It is
natural to ask whether asymmetric fermionic behavior could emerge instead from a
microscopic geometric or topological structure of spacetime itself.

Causal-set theory (CST) provides a minimal approach to discrete Lorentzian
geometry \citep{Bombelli1987, Sorkin2003}. Matter dynamics remain challenging,
but Dirac--Kähler (DK) fermions offer a combinatorial analogue of the continuum
Dirac operator \citep{Rocchini2023}. Prior exploratory work suggested that DK
fermions on random posets might exhibit parity-sensitive behavior. In this work
I perform the first large-scale $3{+}1$-dimensional numerical sweep and show
that a controlled parity bias in the sprinkling distribution produces a robust,
macroscopic chiral index.

The central result is a clear chiral-asymmetry phase diagram in the plane of:
(1) a parity-bias parameter $\varepsilon$, and (2) the Wilson coefficient $r$.
For moderate values of $r$ and sufficiently large $|\varepsilon|$, the index
saturates to $|I| \sim 40$--$50$, consistent across dozens of independent
realizations.

\section{Causal-Set Construction and Parity Bias}

A causal set is generated by a Poisson sprinkling of $N$ points into a bounded
region of $3{+}1$-dimensional Minkowski space \citep{Henson2009}. Points are
time-ordered and causal relations are drawn for timelike separations.

A controlled parity-violating deformation is introduced by rotating each point
$(x,y,z)$ in the $(x,y)$ plane by an angle proportional to a bias parameter
$\varepsilon$:
\begin{equation}
 (x,y,z) \longrightarrow (x\cos\varepsilon - y\sin\varepsilon,\,
                          x\sin\varepsilon + y\cos\varepsilon,\,
                          z).
\end{equation}
This induces a uniform left- or right-handed distortion of the spatial
distribution. No bias is applied in the time coordinate.

The Dirac--Kähler operator $D$ is constructed from oriented incidence matrices
relating vertices, edges, faces, and tetrahedra \citep{Rocchini2023}. A minimal
Wilson term $W(r)$ is included to suppress doubler modes. The chiral index is
defined as
\begin{equation}
  I \equiv n_{+} - n_{-},
\end{equation}
where $n_\pm$ are the number of near-zero eigenvalues of $i\gamma_5 D_{\rm tot}$
with positive/negative sign.

\section{Results}

\subsection{Phase diagram in \texorpdfstring{$(r,\varepsilon)$}{(r,ε)}}

A sweep over $r \in [0.1,0.55]$ and $\varepsilon \in [-1.8,-0.3]$ was performed
with $N=6000$ and $40$ trials per grid point. The resulting chiral-asymmetry
phase diagram is shown in Fig.~\ref{fig:phase}. A large, stable plateau with
$|I| \sim 40$--$50$ appears over most of the parameter range.

\begin{figure}[htbp]
  \centering
  \includegraphics[width=0.95\textwidth]{fig1_phase_diagram.pdf}
  \caption{Phase diagram of the mean chiral index $\langle I\rangle$ as a
  function of Wilson parameter $r$ and parity bias $\varepsilon$, using
  $40$ realizations per grid point. A broad, stable asymmetric phase with
  $|I|\sim 40$--$50$ dominates except near a narrow band around $\varepsilon
  \approx -0.3$.}
  \label{fig:phase}
\end{figure}

\subsection{Slices at fixed \texorpdfstring{$r$}{r}}

Slices at representative $r$ values are shown in Fig.~\ref{fig:slices}. For each
$r$, a sharp crossover in $\varepsilon$ is observed: below a critical value,
$I \approx 0$; above it, the index rapidly saturates. The transition steepens
with increasing $r$.

\begin{figure}[htbp]
  \centering
  \includegraphics[width=0.95\textwidth]{fig2_slices.pdf}
  \caption{Mean chiral index as a function of parity bias $\varepsilon$ for
  selected Wilson coefficients. Error bars represent standard error over
  trials.}
  \label{fig:slices}
\end{figure}

\subsection{Distribution at a representative point}

To illustrate statistical behavior within the asymmetric phase, a histogram of
index values from $40$ trials at a representative point is shown in
Fig.~\ref{fig:hist}. The distribution is strongly skewed, with negligible weight
near $I=0$, confirming that the asymmetry is not a rare-fluctuation artifact.

\begin{figure}[htbp]
  \centering
  \includegraphics[width=0.75\textwidth]{fig3_distribution.pdf}
  \caption{Distribution of the chiral index at a point deep in the asymmetric
  region. The histogram shows a clear shift to negative values with no
  substantial support near zero.}
  \label{fig:hist}
\end{figure}

\section{Discussion}

The emergent chiral index reflects an imbalance of oriented tetrahedra induced
by the parity-biased sprinkling. Because the DK operator directly encodes the
incidence structure of the poset, this imbalance manifests as a genuine
topological effect rather than a numerical artifact. Several features merit
emphasis:

\begin{itemize}
  \item \textbf{Finite-$N$ stability:} The asymmetry persists at $N=6000$ with
        modest fluctuations.
  \item \textbf{Topological origin:} The index depends only on near-zero modes
        of a combinatorial Laplace-type operator.
  \item \textbf{Controlled by parity bias:} The dominant driver is
        $\varepsilon$, with $r$ acting as a smooth regulator via the Wilson
        term.
\end{itemize}

\section{Future Directions}

The present work opens a variety of research avenues:

\begin{itemize}
  \item \textbf{Analytic index structure.} A discrete index theorem for
        Dirac--Kähler fermions on causal sets may be achievable.
  \item \textbf{Finite-size scaling.} A detailed study of the $N$-dependence of
        the transition region could reveal whether a continuum-like critical
        phenomenon exists.
  \item \textbf{Cosmological tests.} Parity-odd vacuum structure such as that
        produced here generically induces parity-odd correlations in the CMB
        (e.g.\ EB/TB mixing). Current constraints are close to the level
        predicted by simple couplings.
  \item \textbf{Gravitational-wave birefringence.} A parity-odd background can
        split the dispersion of the two GW helicities. Reanalyses of LIGO/Virgo
        data with polarization sensitivity provide a promising future test.
\end{itemize}

\section*{Data and Code Availability}

All simulation code, parameters, scripts for generating the figures, and the
full $N=6000$ sweep dataset are available at:\
\url{https://github.com/604Bakker/3-1D-entropic-foam}

A permanent, versioned archive of the exact code and data used in this paper is
available on Zenodo under the concept DOI:\
\url{https://doi.org/10.5281/zenodo.17714411}.


\section*{Acknowledgments}
The author thanks the causal-set research community for discussions and the
open-source scientific Python ecosystem for enabling large-scale computation on
modest hardware.

Portions of code implementation and text polishing benefited from interactive
assistance by large language models (including Grok by xAI and ChatGPT by
OpenAI); all scientific claims and interpretations are the sole responsibility
of the author.


\bibliographystyle{unsrtnat}
\bibliography{references}

\end{document}
